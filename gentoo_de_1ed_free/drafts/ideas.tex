%%%%%%%%%%%%%%%%%%%%%%%%%%%%%%%%%%%%%%%%%%%%
% Ideen f�r evtl. neuauflage des projektes %
%%%%%%%%%%%%%%%%%%%%%%%%%%%%%%%%%%%%%%%%%%%%

% M�gliche weitere Themen:
%  - buildhosts
%  - paludis
%  - lighttpd
%  - virtualisierung

% KAPITEL: tips

% GW: FIXME: Do we want tenshi information?

\subsection{\label{epm}F�r RPM-Umsteiger: \cmd{app-portage/epm}}

% GW: FIXME: Add app-portage/epm

\subsection{\label{demerge}Snapshot des Systems: \cmd{app-portage/demerge}}

% GW: FIXME: Add app-portage/demerge

\section{\label{appportage}Werkzeuge aus der Kategorie \cmd{app-portage}}

Wie der Name der Kategorie verr�t, finden wir hier spezielle
Werkzeuge, die etwas mit dem Paketmanagement-System von Gentoo zu tun
haben.

Einen nicht unerheblichen Teil der Pakete in dieser Klasse haben wir
schon behandelt: \cmd{autounmask} (Kapitel \ref{}), \cmd{eix} (Kapitel
\ref{}), \cmd{emerge-delta-webrsync} (Kapitel \ref{}), \cmd{esearch}
(Kapitel \ref{}), \cmd{euses} (Kapitel \ref{}), \cmd{flagedit}
(Kapitel \ref{}), \cmd{gentoolkit} (Kapitel \ref{}), \cmd{layman}
(Kapitel \ref{}) und \cmd{portage-utils} (Kapitel \ref{}).

Manche der restlichen Pakete sind eher auf die Gentoo-Entwickler
zugeschnitten. Aber die Kategorie bietet noch einige kleine Juwelen,
die in bestimmten Situationen sehr n�tzlich sein k�nnen.

Wir wollen hier noch ein paar der Wichtigeren herausgreifen: \cmd{}
(Kapitel \ref{}), \cmd{} (Kapitel \ref{}), \cmd{} (Kapitel \ref{}),
\cmd{} (Kapitel \ref{}), \cmd{} (Kapitel \ref{})

Grunds�tzlich kann man jedem empfehlen die Kategorie einmal nach
interessanten Werkzeugen zu durchsuchen.

% KAPITEL: paludis

\chapter{Paludis}

Portage ist ein genialer Paket-Manager. Und Portage ist gleichzeitig
eine grauenhafte Software. Um letzteren Standpunkt komme ich aus der
Sicht eines Entwicklers nicht herum. Wer versucht den Python-Code
dieser Applikation zu modifizieren wird sich schnell die Haare
raufen. So gut Portage auch funktionieren mag, es ist eine gewachsene
Software und schleppt einige strukturelle Probleme mit sich.

Ich bin bei weitem nicht der erste Programmierer, dem das
auff�llt. Und entsprechend gibt es derzeit gleich zwei Projekte, die
sich damit besch�ftigen Portage zu ersetzen. Zum einen ist es das
etwas Gentoo-n�here


% ANHANG: rudiment�re Desktop-Konfiguration

\subsection{Tatatureinstellung f�r den X-Server}

Auch wenn es nicht darum geht eine Maschine mit grafischer Oberfl�che
einzurichten, sei doch an dieser Stelle kurz erw�hnt, wie man den
X-Server zur Zusammenarbeit mit einem deutschen Tastaturlayout bringt.

Folgendes geh�rt minimal in die Konfigurationsdatei
\cmd{/etc/X11/xorg.conf}, um den X-Server zur Zusammenarbeit mit einer
deutschen Tastatur zu bewegen:

\begin{ospcode}
Section "InputDevice"

    Identifier	"Keyboard0"
    Driver	"kbd"
    Option      "XkbModel"	"pc102"
    Option      "XkbLayout"	"de"

EndSection
\end{ospcode}

% ANHANG: Dualboot

\appendixchapter{\label{dualboot}Eine Maschine mit Windows teilen}

% GW: FIXME: Add information concerning the DVD boot options.

% GW: FIXME: add information about the cfdisk partitioning tool.


% ANHANG: Interaktiv

% hinweise foren, bug db, irc, web seite, interessante feeds, side project, ...

\section{\label{support}Unterst�tzung finden}

%GW: FIXME: Write

%GW: FIXME: Include emerge --info

% FIXME: Add some comments on the Gentoo speed of development and
% possible problems resulting for this book.

% FIXME: Add some information on forum.gentoo.org, the mailing lists,
% IRC, wiki, the gentoo site in general

%%% Local Variables: 
%%% mode: latex
%%% TeX-master: "gentoo"
%%% coding: latin-1-unix
%%% End: 
